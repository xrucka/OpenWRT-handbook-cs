\section{Předmluva}
Zdravím, tento neoficiální návod k \gls{openwrt} vznikl primárně proto,
aby novým uživatelům pomohl si zvyknout na trochu jiný firmware, než běžně
potkají. Neklade si za cíl povysvětlovat ani naučit vše, ale snad bude
stačit k většině běžných úkonů.

\subsection{Co to je \gls{openwrt}?}
\gls{openwrt} je alternativní firmware pro síťový hardware, založený na
operačním systému Linux. Pokud Linux neznáte a slyšeli jste o něm jen
zkazky, že jde o nějakou zastaralou hrůzu, připomínající největší temnoty
systému MS DOS, pak jim nevěřte. I když to není na první pohled znát, i
takový Android je jen sadou aplikací, spuštěných v Linuxu. Protože -- stejně
stejně jako Linux -- je bezplatně spoluvyvíjen komunitou uživatelů a programátorů,
je množství podporovaných zařízení poněkud omezeno. Naštěstí ale máte jedno
ze zařízení, která podporována jsou.

\subsection{Ovládání}
Co tedy potřebujete vědět o ovládání svého routeru?
\begin{description}
\item{\textbf{Webové rozhraní}}
Webové rozhraní dodávané s \gls{openwrt} se jmenuje \acrshort{luci}
(\acrlong{luci}). Je naprogramováno v jazyce Lua a existuje pro něj řada
rozšiřujících aplikací. Protože je ale \gls{openwrt} čistokrevný Linux
-- a tedy na něm lze provozovat prakticky jakoukoliv Linuxovou aplikaci --
je také současně nemožné naprogramovat webové rozhraní pro každou aplikaci.
\item{\textbf{Terminálové rozhraní}}
Síla (a pro nové uživatele také slabina) ovládání Linuxu spočívá v příkazové
řádce. Na tu se lze připojit pomocí zabezpečené služby \gls{ssh}. Ta probíhá
po šifrovaném komunikačním kanálu, což je jeden ze základních stavebních
kamenů bezpečnosti Vašeho routeru. Protože ale ssh potřebuje pro své fungování
krom uživatelského jména i heslo, není zprvu možné ssh použít. Pro nastavení
hesla tedy použijte webové rozhraní. Pro
\end{description}

\section{První připojení}
Než začnete se zběsilým připojováním routeru k síti, pročtěte si prosím tyto
instrukce. Váš router je po instalaci firmware nastaven tak, aby používal IPv4 adresu
192.168.1.1. Propojte jej tedy síťovým kabelem se svým počítačem a ve webovém
prohlížeči otevřete adresu 192.168.1.1. Klikněte na tlačítko \uv{login} a
přihlašte se.

Po přihlášení Vás uvítá přehledová obrazovka, na které najdete souhrné informace
o Vašem zařízení. Na horní liště najdete nabídku \uv{System}, ve které rozklepněte
položku \uv{Administration}. Tím se dostanete na stránku, kde je možné nastavit
heslo pro správu routeru. Heslo se v žádném případě nevyplatí nechat nenastavené,
avšak pozor - v případě že heslo zapomenete, nebude jednoduché se do routeru
opět dostat. Zvolené heslo si někam pro jistotu poznamenejte a schovejte.
Namísto krkolomných hesel můžete použít jednoduchou větu (nejlépe bez háčků
a čárek), kterou si ovšem snadno zapamatujete.

\subsection{Firmware VDSL}
\label{firstboot:vdslfw}
Nyní tedy ke zprovoznění samotného VDSL.
K tomu budete
potřebovat z internetu stáhnout soubor s firmware\footnote{
\url{https://www.telekom.de/hilfe/downloads/firmware-speedport-w921v-1.40.000.bin}},
%\url{http://hilfe.telekom.de/dlp/eki/downloads/Speedport/Speedport\%20W\%20921V/Firmware\_Speedport\_W921V\_1.40.000.bin}},
% http://www.telekom.de/hilfe/downloads/firmware-speedport-w921v-1.40.000.bin
který z licenčních důvodů není možné dodávat spolu s \gls{openwrt}\footnote{
Je možné, že soubor bude na webu vystaven v jiné verzi -- pak je potřeba tuto
stáhnout a pracovat s přejmenovanou kopií.
}. Soubor stáhněte a nakopírujte programem WinSCP\footnote{Na Linuxu použijte příkaz
{\texttt{scp soubor.bin root@192.168.1.1:/tmp}}.} na svůj router.
Tento soubor uložte na router do adresáře \uv{/tmp}.

Nyní se na přihlaste na router klientem protokolu \gls{ssh} -- pod Windows můžete
použít program \uv{putty}\footnote{\url{http://www.slunecnice.cz/sw/putty-cz/}} -- viz \todo{ilustrace}\footnote{\texttt{ssh root@192.168.1.1}}.
Uživatel pro přihlášení
je \uv{root}, adresa hosta \uv{192.168.1.1}, heslo pak Vámi čerstvě nastavené.
Po přihlášení spusťte následující příkazů, která pozáplatuje instalační skript
pro firmware a následně firmware nainstaluje\footnote{\gls{openwrt} 15.05.1
předpokládá starší, již nedostupnou, verzi firmware.}.
\begin{verbatim}
sed -i 's#Firmware_Speedport_W921V_1.21.000.bin#firmware-speedport-w921v-1.40.000.bin#g' /sbin/vdsl_fw_install.sh
sed -i 's#hilfe.telekom.de/dlp/eki/downloads/Speedport/Speedport%20W%20921V#www.telekom.de/hilfe/downloads#g' /sbin/vdsl_fw_install.sh
sed -i 's#0a099d08dbf091c74d685b532cbb1390#409a69b9a4eeffd681cb2dd84d6edf6d#g' /sbin/vdsl_fw_install.sh
sed -i 's#59dd9dc81195c6854433c691b163f757#655442e31deaa42c9c68944869361ec0#g' /sbin/vdsl_fw_install.sh
sed -i 's#06b6ab3481b8d3eb7e8bf6131f7f6b7f#57f2d07f59e11250ce1219bad99c1eda#g' /sbin/vdsl_fw_install.sh
ls -s /tmp/firmware-speedport-w921v-1.40.000.bin /tmp/Firmware_Speedport_W921V_1.21.000.bin#firmware-speedport-w921v-1.40.000.bin
vdsl_fw_install.sh
\end{verbatim}
Pro rozbalení je potřeba odsouhlasit licenci staženého firmware.

Pro nastavení VDSL připojení nyní postupujte podle návodu v sekci \ref{net:vdsl}.


\todo{Extroot s prvním připojením}
\todo{Failsafe}

