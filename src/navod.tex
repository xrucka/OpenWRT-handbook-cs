\documentclass{scrartcl}
\usepackage[czech]{babel}
\usepackage[utf8]{inputenc}
\usepackage{color}
\usepackage{microtype}
\usepackage{hyphenat}
\usepackage{graphicx}
%\usepackage[toc]{glossaries}
\usepackage{subfigure}
\usepackage{hyperref}
\hypersetup{
    plainpages=false,
    pdfpagelabels,
    bookmarks=true,
    unicode=true,
    pdftoolbar=true,
    pdfmenubar=true,
    % nezapomen aktualizovat odpovidajici klicova slova nize
    colorlinks=true,
    citecolor=black,
    filecolor=black,
    linkcolor=black,
    urlcolor=black
}
\usepackage[nonumberlist]{glossaries}

\definecolor{Orange}{rgb}{1,0.5,0}
\newcommand{\todo}[1]{\textcolor{Orange}{[[TODO: #1]]}}
\definecolor{Red}{rgb}{1,0,0}
\newcommand{\verify}[1]{\textcolor{Red}{[[Verify: #1]]}}
\definecolor{Green}{rgb}{0,1,0}
\newcommand{\comment}[1]{\textcolor{Green}{[COMMENT: #1]}}

\newcommand{\wrtversion}[0]{15.05.1}

\title{Zjednodušený základní návod k~OpenWRT}
\subtitle{pro TP-Link TD-W8970B rev. 1.0}
\author{Lukáš Ručka}
\date{\today}

\renewcommand{\glossarysection}[2][]{}

\newglossaryentry{openwrt}{name={OpenWRT}, description={
OpenWRT je distribuce Linuxu, navržená pro malá zařízení -- mnohdy síťové prvky.
Je sestaveno od základu jako plnohodnotný Linux. To má za následek mimo jiné
to, že je mnohdy aktuálnější (a tedy i bezpečnější) než nejeden oficiální
firmware, dodaný výrobcem či poskytovatelem internetu.
}}
\newglossaryentry{soho}{name={SOHO}, description={
SOHO routery jsou malé routery, navržené pro použití v domácnostech a malých
firmách (Small Office, HOme). Obvykle nekypí kdo ví jak výkonným hardware,
avšak jejich cenová dostupnost z nich dělá velmi rozšířená -- a v případě
některých modelů, kompatibilních s OpenWRT -- i použitelná zařízení.
}}
\newglossaryentry{wan}{name={WAN}, description={
Wide Area Network -- obvyklé označení pro nelokální síť -- zpravidla Internet.
WAN rozhraním jste připojení ke svému poskytovateli internetu (ať už kabelovou
přípojkou, síťovým kabelem nebo VDSL linkou).
}}
\newglossaryentry{lan}{name={LAN}, description={
Local Area Network -- \uv{domácí} síť routeru, resp. síť, pro kterou router
přiděluje adresy a/nebo zprostředkovává připojení k internetu.
}}
\newglossaryentry{vdsl}{name={VDSL}, description={
Very-high-bit-rate Digital Subscriber Line -- technologie v principu rozšiřující
ADSL, ale s jinými signalizacemi. Postupně vytlačuje ADSL.
}}
\newglossaryentry{ssh}{name={SSH}, description={Secure Shell -- protokol
vzdáleného terminálu. Disponuje šifrováním a mechanismem pro přihlašování
bez hesla -- soukromým a veřejným klíčem.
}}
\newglossaryentry{luci}{name={LuCI}, description={Lua Configuration Interface
 -- webové rozhraní pro nastavení OpenWRT.
}}
\newglossaryentry{vlan}{name={VLAN}, description={Virtual LAN - způsob značkování
paketů na fyzické lince, umožňující přenos více logických sítí po stejném médiu.
Číslo logické sítě, nesené v této značce se pak označuje jako VLAN ID.
}}

\makeglossaries

\begin{document}
\maketitle
\section{Předmluva}
Zdravím, tento neoficiální návod k \gls{openwrt} vznikl primárně proto,
aby novým uživatelům pomohl si zvyknout na trochu jiný firmware, než běžně
potkají. Neklade si za cíl povysvětlovat ani naučit vše, ale snad bude
stačit k většině běžných úkonů.

\subsection{Co to je \gls{openwrt}?}
\gls{openwrt} je alternativní firmware pro síťový hardware, založený na
operačním systému Linux. Pokud Linux neznáte a slyšeli jste o něm jen
zkazky, že jde o nějakou zastaralou hrůzu, připomínající největší temnoty
systému MS DOS, pak jim nevěřte. I když to není na první pohled znát, i
takový Android je jen sadou aplikací, spuštěných v Linuxu. Protože -- stejně
stejně jako Linux -- je bezplatně spoluvyvíjen komunitou uživatelů a programátorů,
je množství podporovaných zařízení poněkud omezeno. Naštěstí ale máte jedno
ze zařízení, která podporována jsou.

\subsection{Ovládání}
Co tedy potřebujete vědět o ovládání svého routeru?
\begin{description}
\item{\textbf{Webové rozhraní}}
Webové rozhraní dodávané s \gls{openwrt} se jmenuje \acrshort{luci}
(\acrlong{luci}). Je naprogramováno v jazyce Lua a existuje pro něj řada
rozšiřujících aplikací. Protože je ale \gls{openwrt} čistokrevný Linux
-- a tedy na něm lze provozovat prakticky jakoukoliv Linuxovou aplikaci --
je také současně nemožné naprogramovat webové rozhraní pro každou aplikaci.
\item{\textbf{Terminálové rozhraní}}
Síla (a pro nové uživatele také slabina) ovládání Linuxu spočívá v příkazové
řádce. Na tu se lze připojit pomocí zabezpečené služby \gls{ssh}. Ta probíhá
po šifrovaném komunikačním kanálu, což je jeden ze základních stavebních
kamenů bezpečnosti Vašeho routeru. Protože ale ssh potřebuje pro své fungování
krom uživatelského jména i heslo, není zprvu možné ssh použít. Pro nastavení
hesla tedy použijte webové rozhraní. Pro
\end{description}

\section{První připojení}
Než začnete se zběsilým připojováním routeru k síti, pročtěte si prosím tyto
instrukce. Váš router je po instalaci firmware nastaven tak, aby používal IPv4 adresu
192.168.1.1. Propojte jej tedy síťovým kabelem se svým počítačem a ve webovém
prohlížeči otevřete adresu 192.168.1.1. Klikněte na tlačítko \uv{login} a
přihlašte se.

Po přihlášení Vás uvítá přehledová obrazovka, na které najdete souhrné informace
o Vašem zařízení. Na horní liště najdete nabídku \uv{System}, ve které rozklepněte
položku \uv{Administration}. Tím se dostanete na stránku, kde je možné nastavit
heslo pro správu routeru. Heslo se v žádném případě nevyplatí nechat nenastavené,
avšak pozor - v případě že heslo zapomenete, nebude jednoduché se do routeru
opět dostat. Zvolené heslo si někam pro jistotu poznamenejte a schovejte.
Namísto krkolomných hesel můžete použít jednoduchou větu (nejlépe bez háčků
a čárek), kterou si ovšem snadno zapamatujete.

\subsection{Firmware VDSL}
\label{firstboot:vdslfw}
Nyní tedy ke zprovoznění samotného VDSL.
K tomu budete
potřebovat z internetu stáhnout soubor s firmware\footnote{
\url{https://www.telekom.de/hilfe/downloads/firmware-speedport-w921v-1.40.000.bin}},
%\url{http://hilfe.telekom.de/dlp/eki/downloads/Speedport/Speedport\%20W\%20921V/Firmware\_Speedport\_W921V\_1.40.000.bin}},
% http://www.telekom.de/hilfe/downloads/firmware-speedport-w921v-1.40.000.bin
který z licenčních důvodů není možné dodávat spolu s \gls{openwrt}\footnote{
Je možné, že soubor bude na webu vystaven v jiné verzi -- pak je potřeba tuto
stáhnout a pracovat s přejmenovanou kopií.
}. Soubor stáhněte, změňte číslo verze z 1.40.000 na 1.21.000 \footnote{\gls{openwrt}
\wrtversion počítá se starší verzí firmware VDSL subsystému.}
a nakopírujte programem WinSCP\footnote{Na Linuxu použijte příkaz
{\texttt{scp soubor.bin root@192.168.1.1:/tmp}}.} na svůj router.
Tento soubor uložte na router do adresáře \uv{/tmp}.

Nyní se na přihlaste na router klientem protokolu \gls{ssh} -- pod Windows můžete
použít program \uv{putty}\footnote{\url{http://www.slunecnice.cz/sw/putty-cz/}} -- viz \todo{ilustrace}\footnote{\texttt{ssh root@192.168.1.1}}.
Uživatel pro přihlášení
je \uv{root}, adresa hosta \uv{192.168.1.1}, heslo pak Vámi čerstvě nastavené.
Po přihlášení spusťte příkaz \uv{vdsl\_fw\_install.sh}, který ze staženého firmware
získá firmware pro VDSL subsystém modemu. Pro rozbalení je potřeba odsouhlasit
licenci staženého firmware.

Pro nastavení VDSL připojení nyní postupujte podle návodu v sekci \ref{net:vdsl}.


\todo{Extroot s prvním připojením}
\todo{Failsafe}


\section{První připojení}
Než začnete se zběsilým připojováním routeru k síti, pročtěte si prosím tyto
instrukce. První podsekce slouží jako motivace k externímu úložišti a ačkoliv
není svým obsahem pro úplné začátečníky zcela srozumitelná a odkazuje se na
znalosti z následujících sekcí, nese důležitou
myšlenku. Věnujte jí tedy prosím zvláště zvýšenou pozornost.

Druhá sekce patří také mezi volitelné postupy a stojí za zvážení. Třetí sekce
popíše způsob nastavení hesla a konečně čtvrtá sekce zprovozní VDSL modem
vestavěný do routeru.

Váš router je po instalaci firmware nastaven tak, aby používal IPv4 adresu
192.168.1.1. Propojte jej tedy síťovým kabelem se svým počítačem (využijte
port routeru \uv{Lan 1}).

\subsection{Systém na flashdisku}
\label{firstboot:extroot}
\gls{soho} routery obvykle bývají zařízení s minimem paměti, u kterých výrobce
šetří na každém kousku. Proto se vyplatí využít toho, že \gls{openwrt} umí
využít připojený flashdisk pro rozšíření interní paměti. Toto má ale další
praktický důsledek -- pokud byste si špatným nastavením zablokovali přístup
k routeru, není nic
jednoduššího, než router vypnout, flashdisk připojit do počítače a opravit
nastavení.

Má osobní praxe s každým novým routerem je následující:
\begin{enumerate}
\item Nahraji svůj veřejný
SSH klíč\footnote{Viz např. \url{http://www.root.cz/clanky/jak-se-prihlasovat-na-ssh-bez-zadavani-hesla/}.}.
%\item Router restartuji (vypne službu telnet a povolí \gls{ssh}).
\item Nainstaluji podporu pro externí úložiště.
\item Zakáži \gls{wan} rozhraní.
\item Wifi povolím a přejmenuji na něco dostatečně jednoznačného (např. \uv{router-v-nouzi}).
\item Vypnu přístup přes webové rozhraní.
\item Toto nastavení zkopíruji na flashdisk. Na ten zkopíruji i soubor
\uv{/etc/banner}, kde provedu změnu tak, abych po přihlášení vždy věděl, že
router nastartoval z flashdisku.
\item Router restartuji, ověřím že startuje z flashdisku, povolím webové
rozhraní a nastavím pro ostrý provoz.
\end{enumerate}

Pokud se tedy se systémem na mém routeru cokoliv stane, mohu problém snadno
vyřešit přepsáním flashdisku nejnovější zálohou. Zvláště těm, kdo rádi experimentují
může toto preventivní opatření ušetřit nejednu pernou chvilku.
Tato sekce je záměrně uvedena na začátku návodu k nastavení, neboť pokud
se externí úložiště rozhodnete používat, pak je dobré s ním počítat od úplného
začátku.
Pro detaily
se prosím podívejte na sekci \ref{extroot}.

\subsection{Kryptografické klíče pro SSH}
Ačkoliv není tento krok nutný, je vhodné pokračovat regenerováním klíčů pro
šifrovanou správu zařízení.
Častým neduhem výrobcem dodaného firmware je, že vyrobené routery
sdílejí šifrovací klíče\footnote{\url{http://www.root.cz/clanky/miliony-zarizeni-sdili-privatni-klice-k-https-a-ssh/}}
(což je z hlediska bezpečnosti
přinejmenším nežádoucí). Ačkoliv \gls{openwrt} toto obchází tak, že se klíče
generují až při prvním startu zařízení, je užitečné si klíče vygenerovat znovu
(o chvíli) později ručně.

Nyní se na přihlaste na router pomocí protokolu Telnet\footnote{\gls{openwrt}
ve verzi trunk už Telnet nepodporuje.} -- pod Windows můžete
použít program \uv{putty}\footnote{\url{http://www.slunecnice.cz/sw/putty-cz/}} -- viz \todo{ilustrace}\footnote{\texttt{telnet 192.168.1.1}}.
Zatím není nastaveno heslo a do zařízení se tedy dostanete obratem.
Nyní je potřeba smazat staré klíče a vygenerovat nové. To provedete následujícími
příkazy:
\begin{verbatim}
rm -f /etc/dropbear/*host_key
/etc/init.d/dropbear restart
\end{verbatim}

\subsection{Nastavení hesla}
Než začnete se zběsilým připojováním routeru k síti, pročtěte si prosím tyto
instrukce. Váš router je po instalaci firmware nastaven tak, aby používal IPv4 adresu
192.168.1.1. Propojte jej tedy síťovým kabelem se svým počítačem a ve webovém
prohlížeči otevřete adresu 192.168.1.1. Klikněte na tlačítko \uv{login} a
přihlaste se.

Po přihlášení Vás uvítá přehledová obrazovka, na které najdete souhrnné informace
o Vašem zařízení. Na horní liště najdete nabídku \uv{System}, ve které rozklepněte
položku \uv{Administration}. Tím se dostanete na stránku, kde je možné nastavit
heslo pro správu routeru. Heslo se v žádném případě nevyplatí nechat nenastavené,
avšak pozor - v případě že heslo zapomenete, nebude jednoduché se do routeru
opět dostat. Zvolené heslo si někam pro jistotu poznamenejte a schovejte.
Namísto krkolomných hesel můžete použít jednoduchou větu (nejlépe bez háčků
a čárek), kterou si ovšem snadno zapamatujete.

\subsection{Firmware VDSL}
\label{firstboot:vdslfw}
Nyní tedy ke zprovoznění samotného VDSL.
K tomu budete
potřebovat z internetu stáhnout soubor s firmware\footnote{
\url{https://www.telekom.de/hilfe/downloads/firmware-speedport-w921v-1.40.000.bin}},
%\url{http://hilfe.telekom.de/dlp/eki/downloads/Speedport/Speedport\%20W\%20921V/Firmware\_Speedport\_W921V\_1.40.000.bin}},
% http://www.telekom.de/hilfe/downloads/firmware-speedport-w921v-1.40.000.bin
který z licenčních důvodů není možné dodávat spolu s \gls{openwrt}\footnote{
Je možné, že soubor bude na webu vystaven v jiné verzi -- pak je potřeba tuto
stáhnout a pracovat s přejmenovanou kopií.
}. Soubor stáhněte, změňte číslo verze z 1.40.000 na 1.21.000 \footnote{\gls{openwrt}
\wrtversion počítá se starší verzí firmware VDSL subsystému.}
a nakopírujte programem WinSCP\footnote{Na Linuxu použijte příkaz
{\texttt{scp soubor.bin root@192.168.1.1:/tmp}}.} na svůj router.
Tento soubor uložte na router do adresáře \uv{/tmp}.

Nyní se na přihlaste na router klientem protokolu \gls{ssh} -- pod Windows můžete
použít program \uv{putty}\footnote{\url{http://www.slunecnice.cz/sw/putty-cz/}} -- viz \todo{ilustrace}\footnote{\texttt{ssh root@192.168.1.1}}.
Uživatel pro přihlášení
je \uv{root}, adresa hosta \uv{192.168.1.1}, heslo pak Vámi čerstvě nastavené.
Po přihlášení spusťte příkaz \uv{vdsl\_fw\_install.sh}, který ze staženého firmware
získá firmware pro VDSL subsystém modemu. Pro rozbalení je potřeba odsouhlasit
licenci staženého firmware.

Pro nastavení VDSL připojení nyní postupujte podle návodu v sekci \ref{net:vdsl}.




\section{Jak funguje nastavení}
Nastavení \gls{openwrt} lze (obecně) spravovat dvěma způsoby -- webovým
rozhraním a příkazovou řádkou. Každému uživateli může vyhovovat jiné ovládání,
v obecné rovině ale platí že s příkazovou řádkou zmůžete více. Na základní
ovládání ale plně postačuje webové rozhraní.

Webové rozhraní má dva režimy ukládání změn -- \uv{Save} a \uv{Save \& Apply}.
Režim \uv{Save} pouze poznačí změny v nastavení, avšak neaplikuje je. Pokud
není k dispozici tlačítko \uv{Save \& Apply}, můžete očekávat že změny budou
aplikovány obratem (např. některá nastavení síťových rozhraní). Režim
\uv{Save \& Apply} je pak explicitně rozlišen tam, kde je možné provádět změny
bez jejich okamžité aplikace (\uv{Save}), nebo je naopak žádoucí explicitně
chtít jejich aplikaci (\uv{Save \& Apply}).

Při použití příkazové řádky najdete nastavení základního systému v adresáři
routeru \uv{/etc/config}.
Zde se nachází sada souborů, kde každý soubor odpovídá nějakému subsystému.
S těmito soubory můžete manipulovat buď textovým editorem, nebo pomocí nástroje
\uv{uci}. Soubory jsou dále strukturované na jednotlivé sekce, které obsahují
klíče. Spojením jména subsystému, sekce a klíče pak vzniká plnohodnotný
název klíče. Některé klíče se v téže sekci mohou opakovat, jiné nesmí.

Základní operace s \uv{uci} jsou:
\begin{description}
\item[\textbf{show}] -- příkaz \uv{uci show wireless} vypíše nastavení wifi.
\item[\textbf{set}] -- příkaz \uv{uci set 'wireless.@wifi-iface[0].ssid=OpenWRT-je-doma'}
změní jméno wifi sítě na řetězec \uv{OpenWRT-je-doma}. Apostrofy v příkazu jsou nutné
kvůli speciálním znakům [~] ve jméně klíče. Odpovídá změně uložené tlačítkem
\uv{Save} webového rozhraní.
\item[\textbf{commit}] -- aplikuje změny v nastavení. Odpovídá aplikaci změn,
provedené tlačítkem \uv{Save \& Apply} webového rozhraní.
\item[\textbf{--help}] -- vyobrazí nápovědu k ovládání nástroje uci.
\end{description}

\section{Síť}
Nastavení sítě najdete ve webovém rozhraní v záložce \uv{Network}. Zde,
na kartě \uv{Interfaces} je k dispozici přehled síťových rozhraní, která
jsou na routeru nastavena.

\gls{openwrt} pracuje s konceptem virtuálních rozhraní, která jsou mapována
na rozhraní fyzická. Toto umožňuje jednomu fyzickému rozhraní nastavit více
různých adres. V případě, že nějaké virtuální rozhraní používáme k přidání
adresy, slouží jako jeho fyzické rozhraní jméno virtuálního rozhraní, kterému
přidává adresu. Toto jméno je na začátku předsazeno znakem zavináče. Pro
ilustraci se zaměřte na rozhraní \uv{wan} a \uv{wan6}.

\paragraph{Upozornění:} tlačítko \uv{Delete} provádí operaci \uv{Save \& Apply},
buďte tedy při práci s ním nejvýše opatrní. Pokud budete provádět změny v
nastavení rozhraní LAN, pak si raději vytvořte pomocné rozhraní se statickou
adresou, než abyste si odřízli přistup k routeru.

\subsection{Úprava existujícího rozhraní}
Úpravu existujícího nastavení provedete tlačítkem \uv{edit} u odpovídajícího
síťového rozhraní. Tímto se dostanete k volbám tohoto rozhraní, kde můžete
na záložce \uv{General setup} nastavit IP adresy tohoto rozhraní.

Na kartě \uv{Physical settings} pak nastavujete na která fyzická rozhraní
se má virtuální rozhraní promítat\footnote{Pomocí jména fyzického rozhraní
\uv{@jmenorozhrani} nastavujete doplnění konfigurace virtuálního rozhraní
s názvem \uv{jmenorozhrani}.} Pokud chcete použít stejnou adresaci pro dvě
různá fyzická rozhraní, musí být nad těmito rozhraními vytvořen síťový
bridge. Bridge vytvoří nad síťovým rozhraním virtuální switch, kde
jednotlivá fyzická rozhraní představují porty switche. Toto má za důsledek
to, že počítače z jednoho fyzického rozhraní budou moci komunikovat přímo
s počítači z druhého fyzického rozhraní. Stejně, jako kdyby byly připojeny
do skutečného switche. Řada routerů navíc má vestavěn hardwarový switch,
pokrývající ethernetové porty. Pokud máte ethernetové rozhraní \gls{wan},
nikdy jej nepřidávejte do bridge s \gls{lan} bez ověření, zda nejsou na
stejném hardwarovém switchi.

Na kartě \uv{Firewall} máte k dispozici možnost přiřadit rozhraní do zóny
firewallu. Vlastní firewall pak najdete v samostatné záložce. Pro jeho fungování
je potřeba přijmout, že openwrt používá tzv. zónový firewall - místo definice
pravidel pro jednotlivá fyzická rozhraní definujete pravidla pro zóny.
Konfigurací firewallu tato příručka nepokrývá, s výjimkou některých nastavení
specifických pro síť O2, uvedených na konci příručky.

\subsection{Switch}
\label{net:switch}
V záložce \uv{Network $\rightarrow$ Switch} najdete nastavení vestavěného
hardwarového switche. Tento má tzv. logické porty, jejichž číslování nemusí
odpovídat číslování fyzickému. A mnohdy také neodpovídá. U některých zařízení
jsou LAN a WAN porty prezentovány jako oddělené hardwarové síťové karty
(kde switch je připojen k LAN), na jiných jsou obé jen jinak barevně rozlišené
porty na zádech routeru, všechny připojené do switche. Aby tato rozhraní v
systému vystupovala odděleně, je každému z nich nastavena oddělená \gls{vlan}.
\begin{table}[h!]
\begin{center}
\begin{tabular}{|l|l|}
\hline
Číslo portu switche & fyzické označení portu \\ \hline
0 & LAN 2  \\ \hline
1 & nezapojeno \\ \hline
2 & LAN 3 \\ \hline
3 & nezapojeno \\ \hline
4 & LAN 4 \\ \hline
5 & LAN 1 \\ \hline
6 & CPU \\ \hline
\end{tabular}
\caption{Mapování logických portů switche na fyzické označení portu pro TP-Link TD-W8970B}
\end{center}
\end{table}


Co je to \gls{vlan}? Virtual LAN - způsob značkování
paketů na fyzické lince, umožňující přenos více logických sítí po stejném
médiu. Aby \gls{vlan} fungovala, musí se při vysílání paketu přidat před
paket značka s číslem, do které \gls{vlan} paket patří. Příjemce
pak musí mít pro stejnou \gls{vlan} nastavenu stejnou síť. Protože ne vždy
je toto značkování (angl. tagged) žádoucí, je možné použít neznačkovaný režim
portu. To znamená, že při vysílání ven je značka odstraněna a naopak při
přijetí je značka doplněna (angl. untagged). Port může být veden jako neznačkovaný
pro nejvýše jednu VLAN.

Toto umožňuje realizovat jak WAN, tak LAN rozhraní jedním switchem. Logické
\gls{wan} rozhraní je tvořeno jednou \gls{vlan} (např. VLAN ID 2), kdežto
LAN jinou \gls{vlan} (např. VLAN ID 1). Porty, patřící do fyzické LAN jsou
pak neznačkované (untagged) pro \gls{vlan} 1 se zakázanou \gls{vlan} 2, porty
pro \gls{wan} pak naopak.

Některé routery mají hybridní zapojení síťového hardware -- \gls{wan} i
\gls{lan} jsou přivedeny do stejného switche, kde jsou realizovány pomocí \gls{vlan}.
Avšak namísto toho, aby byl tento switch dalším portem připojen k CPU, je k CPU
připojen hned dvěma porty. Jeden tento port pak patří do \gls{vlan} reprezentující
\gls{wan}, druhý pak plní stejnou úlohu pro \gls{lan}.

Pokud byste chtěli \gls{wan} port přemostit do \gls{lan}, pak si zkontrolujte
že není reprezentován jako \gls{vlan}. Přemostění dvou vlan pomocí bridge
by mohlo vyústit v chybnou konfiguraci routeru. V takovém případě raději přidejte
\gls{wan} port do stejné \gls{vlan}. Naopak, pokud budete chtít některý z LAN
portů použít pro WAN, přiřaďte mu (neznačkovanou) \gls{vlan} použitou pro WAN
port.

Ačkoliv \gls{openwrt} umožňuje mít odlišné VLAN ID a identifikátor pro síťové
rozhraní, je lepší mít tyto nastaveny na stejnou hodnotu -- odlišné nastavení
nemusí na řadě routerů fungovat. Na konkrétní \gls{vlan} se pak V konfiguraci
rozhraní odkážete přes jméno switche (zpravidla \uv{eth0}), doplněné tečkou
a číslem rozhraní (tedy \uv{eth0.1} pro LAN). Aby tato možnost fungovala, musí
být port CPU vždy označen za značkovaný pro každou \gls{vlan}.

Se změnami VLAN buďte opatrní, zvláště pokud zasahujete do připojení k procesoru.
V takovém případě doporučuji jednak použít externí úložiště, druhak nastavit
vše najednou pomocí příkazové řádky (nejlépe textovým editorem), vše si po sobě
několikrát opakovaně přečíst a teprve posléze nastavení aplikovat restartem.

\subsection{VDSL}
\label{net:vdsl}


\section{Systém na externím úložišti (extroot)}
\label{extroot}
Externím úložištěm je zparvidla\footnote{Slovo zpravidla zde stojí oprávněně.
Kupříkladu pro TP-Link LT-MR3220 existuje pěkný návod, jak na nevyužité \gls{gpio}
piny připojit SD kartu (\url{https://wiki.openwrt.org/toh/tp-link/tl-mr3420/deep.mmc.hack}).}
flashdisk, sdkarta či jiné paměťové zařízení, připojené přes USB.

\subsection{Odpojení stávajících souborových stystémů}
\label{extroot:umount}
Než budete pokračovat, musíte zjistit uzel zařízení flashdisku a odpojit
stávající souborové systémy na něm.
\begin{lstlisting}[language=sh]
linux@localhost:/tmp> dmesg | tail | grep Attached # zjistim jmeno uzlu
[167819.765155] sd 7:0:0:0: Attached scsi generic sg4 type 0
[167819.774927] sd 7:0:0:0: [sde] Attached SCSI removable disk
linux@localhost:/tmp> # jmeno uzlu je u mne sde, cesta k zarizeni tedy /dev/sde
linux@localhost:/tmp> # odpojim existujici souborove systemy
linux@localhost:/tmp> sudo umount /dev/sde*
\end{lstlisting}

\subsection{Formátování disku}
Předpokládejme tedy použití flashdisku. Ke zprovoznění USB portu je potřeba
zpravidla balíček {\texttt kmod-usb-storage} a ovladač sběrnice USB (zpravidla
{\texttt kmod-usb2} pro routery osazené porty USB 2.0). Dále je zapotřebí
ovladač zvoleného souborového systému (například {\texttt kmod-fs-ext4}).
Posledními potřebnými balíčky jsou {\texttt block-mount} a {\texttt swap-utils}.

Následující část předpokládá práci s Linuxem, ať už na jiném počítači, nebo
na routeru jako takovém (v takovém případě budete potřebovat i balíčky
{\texttt e2fsprogs fdisk}).

Flashdisk připojte ke zvolenému Linuxu a rozdělte na dva oddíly - první, menší,
bude sloužit jako swap (oddíl, kam se dočasně odkládají nevyužité paměťové
bloky z RAM), druhý jako samotný extroot. Pro menší oddíl bude postačovat
pár MiB (např. 64 MiB), router jej za nomrálních okolností nebude využívat.
ID typu oddílu nastavte na 82 (Linux swap). Druhý oddíl pak zabere zbytek
disku.

Nejprve tedy spustíme program fdisk a na flashdisku vytvoříme prázdnou tabulku
rozložení disku:
\begin{verbatim}
linux@localhost:/tmp> sudo fdisk /dev/sde

Vítejte v fdisku (util-linux 2.28.2).
Změny zůstanou pouze v paměti, dokud se nerozhodnete je uložit na disk.
Při použití příkazu zápisu buďte obezřetní.

Příkaz (m pro nápovědu): o
Vytvořena nová dosová tabulka rozdělení disku s identifikátorem 0x9b990a7e.
\end{verbatim}

Následně připravíme první oddíl:
\begin{verbatim}
Příkaz (m pro nápovědu): n
Typ oddílu
   p   primární (0 primární, 0 rozšířený, 4 volný)
   e   rozšířený (kontejner pro logické oddíly)
Vyberte (výchozí p): p
Číslo oddílu (1-4, výchozí je 1):
První sektor (2048-1965055, výchozí je 2048):
Poslední sektor, +sektorů nebo +velikost{K,M,G,T,P}
	(2048-1965055, výchozí je 1965055): +64M

Vytvořen nový oddíl 1 typu "Linux" o velikosti 64 MiB.

Příkaz (m pro nápovědu): t
Vybrán oddíl 1
Typ oddílu (L vypíše všechny typy): 82
Typ oddílu "Linux" byl změněn na "Linux swap / Solaris".
\end{verbatim}

Nyní druhý oddíl:
\begin{verbatim}
Příkaz (m pro nápovědu): n
Typ oddílu
   p   primární (1 primární, 0 rozšířený, 3 volný)
   e   rozšířený (kontejner pro logické oddíly)
Vyberte (výchozí p):

Použije se výchozí odpověď p.
Číslo oddílu (2-4, výchozí je 2):
První sektor (133120-1965055, výchozí je 133120):
Poslední sektor, +sektorů nebo +velikost{K,M,G,T,P}
	(133120-1965055, výchozí je 1965055):

Vytvořen nový oddíl 2 typu "Linux" o velikosti 894,5 MiB.
\end{verbatim}

Nakonec zkontrolujeme výsledek a zapíšeme tabulku rozložení disku:
\begin{verbatim}
Příkaz (m pro nápovědu): p
Disk /dev/sde: 959,5 MiB, 1 006 108 672 bajtů, 1 965 056 sektorů
Jednotky: sektorů po 1 * 512 = 512 bajtech
Velikost sektoru (logického/fyzického): 512 bajtů / 512 bajtů
Velikost I/O (minimální/optimální): 512 bajtů / 512 bajtů
Typ popisu disku: dos
Identifikátor disku: 0x9b990a7e

Zařízení   Zaveditelný Začátek   Konec Sektory Velikost ID Druh
/dev/sde1                 2048  133119  131072      64M 82 Linux swap/Solaris
/dev/sde2               133120 1965055 1831936   894,5M 83 Linux

Příkaz (m pro nápovědu): w
Tabulka rozdělení disku byla změněna.
Volám ioctl() pro znovunačtení tabulky rozdělení disku.
Synchronizují se disky.
\end{verbatim}

Nově vzniklé oddíly je potřeba inicializovat odpovídajícím souborovým systémem.
Souborový systém extrootu je navíc vhodné pojmenovat, aby nedošlo k náhodnému
přemazání dat potomkem, který hledá flashdisk na svůj powerpoint do školy.
Pokud máte routerů více, je dobré poznačit si do jmenovky, ke kterému routeru
tento disk patří. \gls{uuid} nově vzniklých souborových systémů si poznačte,
budete je potřebovat.
\begin{verbatim}
linux@localhost:/tmp> sudo mkswap /dev/sde1
mkswap: /dev/sde1: pozor: odstraňuje se starý vzorec ext4.
Vytváří se odkládací prostor verze 1, velikost = 64 MiB (67104768 bajtů)
žádná jmenovka, UUID=24623eae-5c3c-4454-86d4-f483a360e39d

linux@localhost:/tmp> sudo mkfs.ext3 -L "openwrt-w8970b/" /dev/sde2
mke2fs 1.43.1 (08-Jun-2016)
/dev/sde2 obsahuje systém souborů ext2 se jmenovkou "ap-extroot""
	naposledy připojeno do /tmp/extroot/overlay v Mon Jan 11 10:38:27 2016
Přesto pokračovat? (a,n) a
Vytváří se systém souborů s 228992 (4k) bloky a 57344 uzly
UUID systému souborů=4a12f341-0046-4aab-aa47-d3baafed85cf
Zálohy superbloku uloženy v blocích:
	32768, 98304, 163840

Alokují se tabulky skupin: hotovo
Zapisuji tabulky iuzlů: hotovo
Vytváří se žurnál (4096 bloků): hotovo
Zapisuji superbloky a účtovací informace systému souborů: hotovo
\end{verbatim}

\subsection{Příprava na straně OpenWRT}
Připravte si kostru souboru {\texttt /etc/config/fstab}. Tento bude obnášet
jednak obecná nastavení, druhak údaje k připojení disků. Z předchozí sekce
máte poznačená UUID, která budete nyní potřebovat. Pokud jste si UUID
nepoznačili, můžete použít nástroj blkid:

\begin{verbatim}
linux@localhost:/tmp> blkid /dev/sde1
/dev/sdd1: UUID="24623eae-5c3c-4454-86d4-f483a360e39d" TYPE="swap" PARTUUID="9b990a7e-01"

linux@localhost:/tmp> blkid /dev/sdd2
/dev/sdd2: UUID="4a12f341-0046-4aab-aa47-d3baafed85cf" SEC_TYPE="ext2" TYPE="ext3" PARTUUID="9b990a7e-02"
\end{verbatim}

Nyní tedy upravme soubor \uv{/etc/config/fstab} tak, aby obsahoval nastavení
uvedená v ilustraci \ref{extroot:fstab}. Nezapomeňte použít \gls{uuid}, která
odpovídají Vašim oddílům.
\begin{figure}
	\lstinputlisting{figure/fstab}
	\caption{Soubor \uv{/etc/config/fstab}}
	\label{extroot:fstab}
\end{figure}

Nyní odpojte flashdisk od stroje, ve kterém jste jej připravovali, a připojte
jej k routeru. Pokud vše funguje správně, měli byste v \uv{/dev} najít uzuly
zařízení.
\begin{verbatim}
root@openwrt:~> ls /dev | grep sd
sda
sda1
sda2
\end{verbatim}

Oddíl extrootu je potřeba připojit a překopírovat na něj obsah adresáře \uv{/overlay}.
\begin{verbatim}
root@openwrt:~> mount /dev/sda2 /tmp/overlay
root@openwrt:~> tar -C /overlay -c . -f - | tar -C /tmp/overlay -xf -
\end{verbatim}

Abychom měli jistotu, že je systém spuštěn z flashdisku, provedeme drobnou
úpravu v souboru \uv{/tmp/overlay/upper/etc/banner}.
\begin{verbatim}
root@openwrt:~> cp /etc/banner /tmp/overlay/upper/etc/banner
root@openwrt:~> echo "extroot online!" >> /tmp/overlay/upper/etc/banner
\end{verbatim}

A to je vše, nyní stačí router restartovat. Po restartu a následném přihlášení
byste měli na konci uvítacího výpisu vidět právě zprávu \uv{extroot online!}.
Pokud ji nevidíte, někde nejspíše došlo k chybě, nejspíše není správně zadáno
\gls{uuid}, nebo nejsou k dispozici uzly zařízení v \uv{/dev}. Oprava už bude
na Vás.

\subsection{Poznámka k (včasnému) použití extrootu}
Jak jsem již popsal v sekci \ref{firstboot:extroot}, vytvoření extrootu jako
jedné z prvních věcí (ještě dříve než nastvím heslo), může jednomu zachránit
spoustu času a úsilí. Správně zabezpečený router je věcí klíčovou pro bezproblémové
fungování sítě. Jste si jisit, že si (doufám že silné) heslo, které pro svůj
router používáte budete pamatovat i za 5 let? Ani sebelepší systém nebude
nikdy dost bezpečný, pokud jeho správce použije heslo, které si sice pamatuje,
ale které je dost slabé aby ho rozlouskl puberťák se základy programování.

\subsection{Import zálohy externího úložiště z obrazu disku}
Obnovení zálohy externího úložiště lze provést pomocí několika jednoduchých
kroků.
\begin{enumerate}
\item Existující obraz zapište na flashdisk, který hodláte používat.
Pokud používáte Windows, budete k tomu potřebovat program Win32 Disk
Imager\footnote{\url{https://sourceforge.net/projects/win32diskimager/}}.
Je možné že tento nástroj bude potřeba spustit s oprávněními správce
(shift + pravé tlačítko myši). Vyberte soubor s obrazem (přípona \uv{.img})
a cílové zařízení. Tlačítkem \uv{Write} pak obraz zapíšete na disk. Pokud
používáte Linux je celá věc mnohem jednodušší, všechny potřebné nástroje
pro zápis již máte v systému. Proveďte tedy přípravu podle \ref{extroot:umount}
a pokračujte níže uvedeným příkazem.
\begin{verbatim}
linux@localhost:/tmp> sudo dd if=OpenWRT-extroot.img of=/dev/sde bs=4K
49408+0 records in
49408+0 records out
202375168 bytes (202 MB, 193 MiB) copied, 40.9951 s, 4.9 MB/s
\end{verbatim}
\item Pro optimální využití kapacity disku je vhodné existující diskové
oddíly zvětšit. Pokud má Váš disk odkládací oddíl (swap), je vhodné jej
nastavit na přibližně dvojnásobek kapacity RAM routeru. Systémový oddíl
můžete nechat vyplnit zbytek volného místa, či jej jen zvětšit a přidat
oddíl pro sdílená data. Takto manipulovat s místem na disku můžete například
pomocí programu GParted\footnote{\url{https://sourceforge.net/projects/gparted/},
Dodáván i formou bootovatelného obrazu.}.
\end{enumerate}

Připojte flashdisk do routeru a přihlaste se na něj. Zkontrolujte, že se
obsah souboru \uv{/etc/config/fstab} odkazuje na správný oddíl na flashdisku.
V případě, že používáte připojení disku pomocí \gls{uuid} můžete identifikátor
oddílu ověřit příkazem:
\begin{verbatim}
blkid /dev/sda2 # jako systemovy slouzi 2. oddil flashdisku
\end{verbatim}
Pokud chcete namísto \gls{uuid} použít cestu k zařízení, použijte klíč
\uv{device} s cestou k zařízení. Po nastavení stačí restartovat router a
ověřit že je \uv{/overlay} připojeno z flashdisku. Kompletní návod lze
nalézt na wiki \gls{openwrt}\footnote{\url{https://wiki.openwrt.org/doc/howto/extroot}}.
\begin{verbatim}
root@openwrt:/sys/class/block# grep overlay /proc/mounts
/dev/sda2 /overlay ext2 rw,relatime 0 0
overlayfs:/overlay / overlay rw,noatime,lowerdir=/,upperdir=/overlay/upper,workdir=/overlay/work 0 0
\end{verbatim}

\input{software}
\section{Nastavení specifická pro 02}
\subsection{VDSL}
O2 používá pro svou síť jednotné přihlašovací údaje -- protokol \uv{pppoe},
uživatel \uv{o2}, heslo \uv{o2}. \gls{vdsl} rozhraní pak odpovídající ptm
(pokud nemá router více modemů zpravidla \uv{ptm0}), VLAN ID 848 (\uv{ptm0.848}).
Protokol získání IPv4 adresy DHCP. IPv6 pak skrze rozhraní wan6 svázané s \uv{@wan},
protokol DHCPv6.

Pro IPTV je pak potřeba použít bridge, do kterého jsou připojeny porty k
settopboxům a vdsl rozhraní \uv{ptm0.835}.

\subsection{Firewall}
\todo{dhcpv6, iptv}

\subsection{Systém}
Pro správnou funkci je potřeba zakázat IGMP snooping při startu routeru. Toho
docílíte následujícím řádkem, umístěným do souboru \uv{/etc/rc.local}: \uv{echo "0" > /sys/devices/virtual/net/br-IPTV/bridge/multicast\_snooping}
\todo{listings}

\subsection{Úryvky konfiguračních souborů}
\todo{rc.local}
\todo{config/network}
\todo{config/firewall}

\end{document}
