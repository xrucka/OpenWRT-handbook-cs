\subsection{Nastavení hesla}
Než začnete se zběsilým připojováním routeru k síti, pročtěte si prosím tyto
instrukce. Váš router je po instalaci firmware nastaven tak, aby používal IPv4 adresu
192.168.1.1. Propojte jej tedy síťovým kabelem se svým počítačem a ve webovém
prohlížeči otevřete adresu 192.168.1.1. Klikněte na tlačítko \uv{login} a
přihlaste se.

Po přihlášení Vás uvítá přehledová obrazovka, na které najdete souhrnné informace
o Vašem zařízení. Na horní liště najdete nabídku \uv{System}, ve které rozklepněte
položku \uv{Administration}. Tím se dostanete na stránku, kde je možné nastavit
heslo pro správu routeru. Heslo se v žádném případě nevyplatí nechat nenastavené,
avšak pozor - v případě že heslo zapomenete, nebude jednoduché se do routeru
opět dostat. Zvolené heslo si někam pro jistotu poznamenejte a schovejte.
Namísto krkolomných hesel můžete použít jednoduchou větu (nejlépe bez háčků
a čárek), kterou si ovšem snadno zapamatujete.
