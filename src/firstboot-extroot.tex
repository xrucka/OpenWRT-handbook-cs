\subsection{Systém na flashdisku}
\label{firstboot:extroot}
\gls{soho} routery obvykle bývají zařízení s minimem paměti, u kterých výrobce
šetří na každém kousku. Proto se vyplatí využít toho, že \gls{openwrt} umí
využít připojený flashdisk pro rozšíření interní paměti. Toto má ale další
praktický důsledek -- pokud byste si špatným nastavením zablokovali přístup
k routeru, není nic
jednoduššího, než router vypnout, flashdisk připojit do počítače a opravit
nastavení.

Má osobní praxe s každým novým routerem je následující:
\begin{enumerate}
\item Nahraji svůj veřejný
SSH klíč\footnote{Viz např. \url{http://www.root.cz/clanky/jak-se-prihlasovat-na-ssh-bez-zadavani-hesla/}.}.
%\item Router restartuji (vypne službu telnet a povolí \gls{ssh}).
\item Nainstaluji podporu pro externí úložiště.
\item Zakáži \gls{wan} rozhraní.
\item Wifi povolím a přejmenuji na něco dostatečně jednoznačného (např. \uv{router-v-nouzi}).
\item Vypnu přístup přes webové rozhraní.
\item Toto nastavení zkopíruji na flashdisk. Na ten zkopíruji i soubor
\uv{/etc/banner}, kde provedu změnu tak, abych po přihlášení vždy věděl, že
router nastartoval z flashdisku.
\item Router restartuji, ověřím že startuje z flashdisku, povolím webové
rozhraní a nastavím pro ostrý provoz.
\end{enumerate}

Pokud se tedy se systémem na mém routeru cokoliv stane, mohu problém snadno
vyřešit přepsáním flashdisku nejnovější zálohou. Zvláště těm, kdo rádi experimentují
může toto preventivní opatření ušetřit nejednu pernou chvilku.
Tato sekce je záměrně uvedena na začátku návodu k nastavení, neboť pokud
se externí úložiště rozhodnete používat, pak je dobré s ním počítat od úplného
začátku.
Pro detaily
se prosím podívejte na sekci \ref{extroot}.
