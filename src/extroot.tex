\section{Systém na externím úložišti (extroot)}
\subsection{Vytvoření externího úložiště}
\todo{Rozepsat - partice, souborový systém, vysvětlit UUID}

\subsection{Import zálohy externího úložiště}
Obnovení zálohy externího úložiště lze provést pomocí několika jednoduchých
kroků.
\begin{enumerate}
\item Existující obraz zapište na flashdisk, který hodláte používat.
Pokud používáte Windows, budete k tomu potřebovat program Win32 Disk
Imager\footnote{\url{https://sourceforge.net/projects/win32diskimager/}}.
Je možné že tento nástroj bude potřeba spustit s oprávněními správce
(shift + pravé tlačítko myši). Vyberte soubor s obrazem (přípona \uv{.img})
a cílové zařízení. Tlačítkem \uv{Write} pak obraz zapíšete na disk. Pokud
používáte Linux je celá věc mnohem jednodušší, všechny potřebné nástroje
pro zápis již máte v systému. Vše co potřebujete znát je pouze uzel zařízení,
na které budete zapisovat.
\begin{lstlisting}[language=sh]
linux@localhost:/tmp> dmesg | tail | grep Attached # zjistim jmeno uzlu
[167819.765155] sd 7:0:0:0: Attached scsi generic sg4 type 0
[167819.774927] sd 7:0:0:0: [sde] Attached SCSI removable disk
linux@localhost:/tmp> # jmeno uzlu je u mne sde, cesta k zarizeni tedy /dev/sde
linux@localhost:/tmp> # odpojim existujici souborove systemy
linux@localhost:/tmp> sudo umount /dev/sde*
linux@localhost:/tmp> sudo dd if=OpenWRT-extroot.img of=/dev/sde bs=4K
49408+0 records in
49408+0 records out
202375168 bytes (202 MB, 193 MiB) copied, 40.9951 s, 4.9 MB/s
\end{lstlisting}
\item Pro optimální využití kapacity disku je vhodné existující diskové
oddíly zvětšit. Pokud má Váš disk odkládací oddíl (swap), je vhodné jej
nastavit na přibližně dvojnásobek kapacity RAM routeru. Systémový oddíl
můžete nechat vyplnit zbytek volného místa, či jej jen zvětšit a přidat
oddíl pro sdílená data. Takto manipulovat s místem na disku můžete například
pomocí programu GParted\footnote{\url{https://sourceforge.net/projects/gparted/},
Dodáván i formou bootovatelného obrazu.}.
\end{enumerate}

Připojte flashdisk do routeru a přihlaste se na něj. Zkontrolujte, že se
obsah souboru \uv{/etc/config/fstab} odkazuje na správný oddíl na flashdisku.
V případě, že používáte připojení disku pomocí \gls{uuid} můžete identifikátor
oddílu ověřit příkazem:
\begin{lstlisting}[language=sh]
blkid /dev/sda2 # jako systemovy slouzi 2. oddil flashdisku
\end{lstlisting}
Pokud chcete namísto \gls{uuid} použít cestu k zařízení, použijte klíč
\uv{device} s cestou k zařízení. Po nastavení stačí restartovat router a
ověřit že je \uv{/overlay} připojeno z flashdisku. Kompletní návod lze
nalézt na wiki \gls{openwrt}\footnote{\url{https://wiki.openwrt.org/doc/howto/extroot}}.
\begin{lstlisting}[language=sh]
root@openwrt:/sys/class/block# grep overlay /proc/mounts
/dev/sda2 /overlay ext2 rw,relatime 0 0
overlayfs:/overlay / overlay rw,noatime,lowerdir=/,upperdir=/overlay/upper,workdir=/overlay/work 0 0
\end{lstlisting}
