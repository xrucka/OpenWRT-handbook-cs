\section{Systém na externím úložišti (extroot)}
\label{extroot}
Externím úložištěm je zparvidla\footnote{Slovo zpravidla zde stojí oprávněně.
Kupříkladu pro TP-Link LT-MR3220 existuje pěkný návod, jak na nevyužité \gls{gpio}
piny připojit SD kartu (\url{https://wiki.openwrt.org/toh/tp-link/tl-mr3420/deep.mmc.hack}).}
flashdisk, sdkarta či jiné paměťové zařízení, připojené přes USB.

\subsection{Odpojení stávajících souborových stystémů}
\label{extroot:umount}
Než budete pokračovat, musíte zjistit uzel zařízení flashdisku a odpojit
stávající souborové systémy na něm.
\begin{lstlisting}[language=sh]
linux@localhost:/tmp> dmesg | tail | grep Attached # zjistim jmeno uzlu
[167819.765155] sd 7:0:0:0: Attached scsi generic sg4 type 0
[167819.774927] sd 7:0:0:0: [sde] Attached SCSI removable disk
linux@localhost:/tmp> # jmeno uzlu je u mne sde, cesta k zarizeni tedy /dev/sde
linux@localhost:/tmp> # odpojim existujici souborove systemy
linux@localhost:/tmp> sudo umount /dev/sde*
\end{lstlisting}

\subsection{Formátování disku}
Předpokládejme tedy použití flashdisku. Ke zprovoznění USB portu je potřeba
zpravidla balíček {\texttt kmod-usb-storage} a ovladač sběrnice USB (zpravidla
{\texttt kmod-usb2} pro routery osazené porty USB 2.0). Dále je zapotřebí
ovladač zvoleného souborového systému (například {\texttt kmod-fs-ext4}).
Posledními potřebnými balíčky jsou {\texttt block-mount} a {\texttt swap-utils}.

Následující část předpokládá práci s Linuxem, ať už na jiném počítači, nebo
na routeru jako takovém (v takovém případě budete potřebovat i balíčky
{\texttt e2fsprogs fdisk}).

Flashdisk připojte ke zvolenému Linuxu a rozdělte na dva oddíly - první, menší,
bude sloužit jako swap (oddíl, kam se dočasně odkládají nevyužité paměťové
bloky z RAM), druhý jako samotný extroot. Pro menší oddíl bude postačovat
pár MiB (např. 64 MiB), router jej za nomrálních okolností nebude využívat.
ID typu oddílu nastavte na 82 (Linux swap). Druhý oddíl pak zabere zbytek
disku.

Nejprve tedy spustíme program fdisk a na flashdisku vytvoříme prázdnou tabulku
rozložení disku:
\begin{verbatim}
linux@localhost:/tmp> sudo fdisk /dev/sde

Vítejte v fdisku (util-linux 2.28.2).
Změny zůstanou pouze v paměti, dokud se nerozhodnete je uložit na disk.
Při použití příkazu zápisu buďte obezřetní.

Příkaz (m pro nápovědu): o
Vytvořena nová dosová tabulka rozdělení disku s identifikátorem 0x9b990a7e.
\end{verbatim}

Následně připravíme první oddíl:
\begin{verbatim}
Příkaz (m pro nápovědu): n
Typ oddílu
   p   primární (0 primární, 0 rozšířený, 4 volný)
   e   rozšířený (kontejner pro logické oddíly)
Vyberte (výchozí p): p
Číslo oddílu (1-4, výchozí je 1):
První sektor (2048-1965055, výchozí je 2048):
Poslední sektor, +sektorů nebo +velikost{K,M,G,T,P}
	(2048-1965055, výchozí je 1965055): +64M

Vytvořen nový oddíl 1 typu "Linux" o velikosti 64 MiB.

Příkaz (m pro nápovědu): t
Vybrán oddíl 1
Typ oddílu (L vypíše všechny typy): 82
Typ oddílu "Linux" byl změněn na "Linux swap / Solaris".
\end{verbatim}

Nyní druhý oddíl:
\begin{verbatim}
Příkaz (m pro nápovědu): n
Typ oddílu
   p   primární (1 primární, 0 rozšířený, 3 volný)
   e   rozšířený (kontejner pro logické oddíly)
Vyberte (výchozí p):

Použije se výchozí odpověď p.
Číslo oddílu (2-4, výchozí je 2):
První sektor (133120-1965055, výchozí je 133120):
Poslední sektor, +sektorů nebo +velikost{K,M,G,T,P}
	(133120-1965055, výchozí je 1965055):

Vytvořen nový oddíl 2 typu "Linux" o velikosti 894,5 MiB.
\end{verbatim}

Nakonec zkontrolujeme výsledek a zapíšeme tabulku rozložení disku:
\begin{verbatim}
Příkaz (m pro nápovědu): p
Disk /dev/sde: 959,5 MiB, 1 006 108 672 bajtů, 1 965 056 sektorů
Jednotky: sektorů po 1 * 512 = 512 bajtech
Velikost sektoru (logického/fyzického): 512 bajtů / 512 bajtů
Velikost I/O (minimální/optimální): 512 bajtů / 512 bajtů
Typ popisu disku: dos
Identifikátor disku: 0x9b990a7e

Zařízení   Zaveditelný Začátek   Konec Sektory Velikost ID Druh
/dev/sde1                 2048  133119  131072      64M 82 Linux swap/Solaris
/dev/sde2               133120 1965055 1831936   894,5M 83 Linux

Příkaz (m pro nápovědu): w
Tabulka rozdělení disku byla změněna.
Volám ioctl() pro znovunačtení tabulky rozdělení disku.
Synchronizují se disky.
\end{verbatim}

Nově vzniklé oddíly je potřeba inicializovat odpovídajícím souborovým systémem.
Souborový systém extrootu je navíc vhodné pojmenovat, aby nedošlo k náhodnému
přemazání dat potomkem, který hledá flashdisk na svůj powerpoint do školy.
Pokud máte routerů více, je dobré poznačit si do jmenovky, ke kterému routeru
tento disk patří. \gls{uuid} nově vzniklých souborových systémů si poznačte,
budete je potřebovat.
\begin{verbatim}
linux@localhost:/tmp> sudo mkswap /dev/sde1
mkswap: /dev/sde1: pozor: odstraňuje se starý vzorec ext4.
Vytváří se odkládací prostor verze 1, velikost = 64 MiB (67104768 bajtů)
žádná jmenovka, UUID=24623eae-5c3c-4454-86d4-f483a360e39d

linux@localhost:/tmp> sudo mkfs.ext3 -L "openwrt-w8970b/" /dev/sde2
mke2fs 1.43.1 (08-Jun-2016)
/dev/sde2 obsahuje systém souborů ext2 se jmenovkou "ap-extroot""
	naposledy připojeno do /tmp/extroot/overlay v Mon Jan 11 10:38:27 2016
Přesto pokračovat? (a,n) a
Vytváří se systém souborů s 228992 (4k) bloky a 57344 uzly
UUID systému souborů=4a12f341-0046-4aab-aa47-d3baafed85cf
Zálohy superbloku uloženy v blocích:
	32768, 98304, 163840

Alokují se tabulky skupin: hotovo
Zapisuji tabulky iuzlů: hotovo
Vytváří se žurnál (4096 bloků): hotovo
Zapisuji superbloky a účtovací informace systému souborů: hotovo
\end{verbatim}

\subsection{Příprava na straně OpenWRT}
Připravte si kostru souboru {\texttt /etc/config/fstab}. Tento bude obnášet
jednak obecná nastavení, druhak údaje k připojení disků. Z předchozí sekce
máte poznačená UUID, která budete nyní potřebovat. Pokud jste si UUID
nepoznačili, můžete použít nástroj blkid:

\begin{verbatim}
linux@localhost:/tmp> blkid /dev/sde1
/dev/sdd1: UUID="24623eae-5c3c-4454-86d4-f483a360e39d" TYPE="swap" PARTUUID="9b990a7e-01"

linux@localhost:/tmp> blkid /dev/sdd2
/dev/sdd2: UUID="4a12f341-0046-4aab-aa47-d3baafed85cf" SEC_TYPE="ext2" TYPE="ext3" PARTUUID="9b990a7e-02"
\end{verbatim}

Nyní tedy upravme soubor \uv{/etc/config/fstab} tak, aby obsahoval nastavení
uvedená v ilustraci \ref{extroot:fstab}. Nezapomeňte použít \gls{uuid}, která
odpovídají Vašim oddílům.
\begin{figure}
	\lstinputlisting{figure/fstab}
	\caption{Soubor \uv{/etc/config/fstab}}
	\label{extroot:fstab}
\end{figure}

Nyní odpojte flashdisk od stroje, ve kterém jste jej připravovali, a připojte
jej k routeru. Pokud vše funguje správně, měli byste v \uv{/dev} najít uzuly
zařízení.
\begin{verbatim}
root@openwrt:~> ls /dev | grep sd
sda
sda1
sda2
\end{verbatim}

Oddíl extrootu je potřeba připojit a překopírovat na něj obsah adresáře \uv{/overlay}.
\begin{verbatim}
root@openwrt:~> mount /dev/sda2 /tmp/overlay
root@openwrt:~> tar -C /overlay -c . -f - | tar -C /tmp/overlay -xf -
\end{verbatim}

Abychom měli jistotu, že je systém spuštěn z flashdisku, provedeme drobnou
úpravu v souboru \uv{/tmp/overlay/upper/etc/banner}.
\begin{verbatim}
root@openwrt:~> cp /etc/banner /tmp/overlay/upper/etc/banner
root@openwrt:~> echo "extroot online!" >> /tmp/overlay/upper/etc/banner
\end{verbatim}

A to je vše, nyní stačí router restartovat. Po restartu a následném přihlášení
byste měli na konci uvítacího výpisu vidět právě zprávu \uv{extroot online!}.
Pokud ji nevidíte, někde nejspíše došlo k chybě, nejspíše není správně zadáno
\gls{uuid}, nebo nejsou k dispozici uzly zařízení v \uv{/dev}. Oprava už bude
na Vás.

\subsection{Poznámka k (včasnému) použití extrootu}
Jak jsem již popsal v sekci \ref{firstboot:extroot}, vytvoření extrootu jako
jedné z prvních věcí (ještě dříve než nastvím heslo), může jednomu zachránit
spoustu času a úsilí. Správně zabezpečený router je věcí klíčovou pro bezproblémové
fungování sítě. Jste si jisit, že si (doufám že silné) heslo, které pro svůj
router používáte budete pamatovat i za 5 let? Ani sebelepší systém nebude
nikdy dost bezpečný, pokud jeho správce použije heslo, které si sice pamatuje,
ale které je dost slabé aby ho rozlouskl puberťák se základy programování.

\subsection{Import zálohy externího úložiště z obrazu disku}
Obnovení zálohy externího úložiště lze provést pomocí několika jednoduchých
kroků.
\begin{enumerate}
\item Existující obraz zapište na flashdisk, který hodláte používat.
Pokud používáte Windows, budete k tomu potřebovat program Win32 Disk
Imager\footnote{\url{https://sourceforge.net/projects/win32diskimager/}}.
Je možné že tento nástroj bude potřeba spustit s oprávněními správce
(shift + pravé tlačítko myši). Vyberte soubor s obrazem (přípona \uv{.img})
a cílové zařízení. Tlačítkem \uv{Write} pak obraz zapíšete na disk. Pokud
používáte Linux je celá věc mnohem jednodušší, všechny potřebné nástroje
pro zápis již máte v systému. Proveďte tedy přípravu podle \ref{extroot:umount}
a pokračujte níže uvedeným příkazem.
\begin{verbatim}
linux@localhost:/tmp> sudo dd if=OpenWRT-extroot.img of=/dev/sde bs=4K
49408+0 records in
49408+0 records out
202375168 bytes (202 MB, 193 MiB) copied, 40.9951 s, 4.9 MB/s
\end{verbatim}
\item Pro optimální využití kapacity disku je vhodné existující diskové
oddíly zvětšit. Pokud má Váš disk odkládací oddíl (swap), je vhodné jej
nastavit na přibližně dvojnásobek kapacity RAM routeru. Systémový oddíl
můžete nechat vyplnit zbytek volného místa, či jej jen zvětšit a přidat
oddíl pro sdílená data. Takto manipulovat s místem na disku můžete například
pomocí programu GParted\footnote{\url{https://sourceforge.net/projects/gparted/},
Dodáván i formou bootovatelného obrazu.}.
\end{enumerate}

Připojte flashdisk do routeru a přihlaste se na něj. Zkontrolujte, že se
obsah souboru \uv{/etc/config/fstab} odkazuje na správný oddíl na flashdisku.
V případě, že používáte připojení disku pomocí \gls{uuid} můžete identifikátor
oddílu ověřit příkazem:
\begin{verbatim}
blkid /dev/sda2 # jako systemovy slouzi 2. oddil flashdisku
\end{verbatim}
Pokud chcete namísto \gls{uuid} použít cestu k zařízení, použijte klíč
\uv{device} s cestou k zařízení. Po nastavení stačí restartovat router a
ověřit že je \uv{/overlay} připojeno z flashdisku. Kompletní návod lze
nalézt na wiki \gls{openwrt}\footnote{\url{https://wiki.openwrt.org/doc/howto/extroot}}.
\begin{verbatim}
root@openwrt:/sys/class/block# grep overlay /proc/mounts
/dev/sda2 /overlay ext2 rw,relatime 0 0
overlayfs:/overlay / overlay rw,noatime,lowerdir=/,upperdir=/overlay/upper,workdir=/overlay/work 0 0
\end{verbatim}
