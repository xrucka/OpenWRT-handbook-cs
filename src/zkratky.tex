\newglossaryentry{openwrt}{name={OpenWRT}, description={
OpenWRT je distribuce Linuxu, navržená pro malá zařízení -- mnohdy síťové prvky.
Je sestaveno od základu jako plnohodnotný Linux. To má za následek mimo jiné
to, že je mnohdy aktuálnější (a tedy i bezpečnější) než nejeden oficiální
firmware, dodaný výrobcem či poskytovatelem internetu.
}}
\newglossaryentry{soho}{name={SOHO}, description={
SOHO routery jsou malé routery, navržené pro použití v domácnostech a malých
firmách (Small Office, HOme). Obvykle nekypí kdo ví jak výkonným hardware,
avšak jejich cenová dostupnost z nich dělá velmi rozšířená -- a v případě
některých modelů, kompatibilních s OpenWRT -- i použitelná zařízení.
}}
\newglossaryentry{wan}{name={WAN}, description={
Wide Area Network -- obvyklé označení pro nelokální síť -- zpravidla Internet.
WAN rozhraním jste připojení ke svému poskytovateli internetu (ať už kabelovou
přípojkou, síťovým kabelem nebo VDSL linkou).
}}
\newglossaryentry{lan}{name={LAN}, description={
Local Area Network -- \uv{domácí} síť routeru, resp. síť, pro kterou router
přiděluje adresy a/nebo zprostředkovává připojení k internetu.
}}
\newglossaryentry{vdsl}{name={VDSL}, description={
Very-high-bit-rate Digital Subscriber Line -- technologie v principu rozšiřující
ADSL, ale s jinými signalizacemi. Postupně vytlačuje ADSL.
}}
\newglossaryentry{adsl}{name={ADSL}, description={
Asymmetric Digital Subscriber Line -- starší typ DSL, dnes postupně vytlačován
VDSL.
}}
\newglossaryentry{dslam}{name={DSLAM}, description={
Digital Subscriber Line Access Multiplexer -- zjednodušeně řečeno, protikus xDSL modemu.
}}
\newglossaryentry{soc}{name={SOC}, description={
System on Chip představuje spojení čipové sady základní desky a procesoru,
zpravidla do jednoho či několika málo čipů. Právě volba SOC bývá určující
pro výkon routeru a podporované funkce.
}}
\newglossaryentry{ssh}{name={SSH}, description={Secure Shell -- protokol
vzdáleného terminálu. Disponuje šifrováním a mechanismem pro přihlašování
bez hesla -- soukromým a veřejným klíčem.
}}
\newglossaryentry{luci}{name={LuCI}, description={Lua Configuration Interface
 -- webové rozhraní pro nastavení OpenWRT.
}}
\newglossaryentry{vlan}{name={VLAN}, description={Virtual LAN - způsob značkování
paketů na fyzické lince, umožňující přenos více logických sítí po stejném médiu.
Číslo logické sítě, nesené v této značce se pak označuje jako VLAN ID.
}}
\newglossaryentry{uuid}{name={UUID}, description={
Universally Unique IDentifier -- unikátní identifikátor, zpravidla disku.
Narozdíl od uzlu zařízení nezávisí na tom, kdy bylo zařízení připojeno a
je přenositelný mezi počítači.
}}
\newglossaryentry{gpio}{name={GPIO}, description={
General Purpose Input Output -- sada pinů, které lze použít k propojení s
takřka libovolným hardware. Za obecnost platí tím, že řadič takto vytvořeného
propojení je nutno implementovat v software.
}}
