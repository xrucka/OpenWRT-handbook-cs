\subsection{Kryptografické klíče pro SSH}
Ačkoliv není tento krok nutný, je vhodné pokračovat regenerováním klíčů pro
šifrovanou správu zařízení.
Častým neduhem výrobcem dodaného firmware je, že vyrobené routery
sdílejí šifrovací klíče\footnote{\url{http://www.root.cz/clanky/miliony-zarizeni-sdili-privatni-klice-k-https-a-ssh/}}
(což je z hlediska bezpečnosti
přinejmenším nežádoucí). Ačkoliv \gls{openwrt} toto obchází tak, že se klíče
generují až při prvním startu zařízení, je užitečné si klíče vygenerovat znovu
(o chvíli) později ručně.

Nyní se na přihlaste na router pomocí protokolu Telnet\footnote{\gls{openwrt}
ve verzi trunk už Telnet nepodporuje.} -- pod Windows můžete
použít program \uv{putty}\footnote{\url{http://www.slunecnice.cz/sw/putty-cz/}} -- viz \todo{ilustrace}\footnote{\texttt{telnet 192.168.1.1}}.
Zatím není nastaveno heslo a do zařízení se tedy dostanete obratem.
Nyní je potřeba smazat staré klíče a vygenerovat nové. To provedete následujícími
příkazy:
\begin{verbatim}
rm -f /etc/dropbear/*host_key
/etc/init.d/dropbear restart
\end{verbatim}
