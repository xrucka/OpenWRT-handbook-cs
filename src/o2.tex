\section{Nastavení specifická pro 02}
\subsection{IPTV}
Settopboxy pro IPTV, které O2 distribuuje přebírají živé vysílání z multicastu,
program pak podpůrným protokolem (zdá se že jde o http, blíže jsem toto nezkoumal).
Nahrané pořady jsou pak neseny po UDP jako unicast. Settopbox tedy očekává
že je připojen do sítě O2 v režimu bridge\footnote{V modemu Zyxel VMG1312 B30B,
na němž jsem svá nastavení zakládal, měl bridge přiřazenu IPv4 adresu 192.168.1.2.
Toto odrážím i ve své konfiguraci, ačkoliv mám za to že je to zcela zbytečné.}.
Tento mám ve svých nastaveních
zaveden jako IPTV. Jedním virtuálním zařízením do něj připojeným je
odpovídající uplink (viz následující sekce), druhým pak dedikovaná VLAN
na switchi. Porty této VLAN jsou vedeny jako neznačkované.

\subsection{VDSL}
O2 používá pro svou síť jednotné přihlašovací údaje -- protokol \uv{pppoe},
uživatel \uv{o2}, heslo \uv{o2}. \gls{vdsl} rozhraní pak odpovídající ptm
(pokud nemá router více modemů zpravidla \uv{ptm0}), VLAN ID 848 (\uv{ptm0.848}).
Protokol získání IPv4 adresy DHCP. IPv6 pak skrze rozhraní wan6 svázané s \uv{@wan},
protokol DHCPv6.

Pro IPTV je pak potřeba použít bridge, do kterého je připojena vyhrazená VLAN
pro multicast (vedoucí k settopboxům) a vdsl rozhraní \uv{ptm0.835}.

\subsection{Firewall}
O2 přiděluje IPv6 adresy DHCPv6 serverem, který naslouchá na požadavky na
link-local adrese, avšak odpovídá svou globální veřejnou adresou. Je tedy potřeba
povolit provoz ze systémové sítě 02 \uv{2a00:1028:1:910::1/48}, zdrojového
UDP portu 547 na cílový port 546. Na stejné adrese naslouchá i DNS server,
je tedy žádoucí přidat další pravidlo pro příchozí odpovědi se zdrojovým
portem 53 (cílový port nespecifikován, cílová adresa libovolná). Posledním
pravidlem pro tuto adresu je pak povolení příchozích ICMPv6 zpráv.

Při zprovozňování routeru v síti O2 jsem narazil na jedno nepříliš milé
překvapení -- vestavěná pravidla \uv{Allow-ICMPv6-Input} a \uv{Allow-DHCPv6}
nefungovala korektně -- nebyla schopna dojít párování k zóně. Proto používám
jejich verzi bez označené zóny (pojmenována \uv{link-local DHCPv6} a
\uv{link-local ICMPv6}), jde ale s největší pravděpodobností o chybu
někde v mém nastavení.

Poslední sada pravidel se týká IPTV. Krom nastavení pravidel zóny samotné
používám pro klid duše duplicitní sadu pravidel. Nastavení zóny je prosté
-- rozhodně nechceme, aby byly jakékoliv packety z bridge pro IPTV směrovány
do wan, ani do naší vnitřní sítě či určeny modemu jako takovému -- zóně tedy
zakážeme jak přesměrování, tak vstup (forward a input). Zakázat lze dvěma
způsoby -- \uv{deny} a \uv{drop}. Deny generuje odpovídající hlášení o zakázaném
provozu, které je zasláno protistraně. Drop paket jednoduše zahodí. Osobně
doporučuji pro toto zapojení použít drop. Dodatečná duplicitní pravidla pak
zakazují jakýkoliv provoz IPTV$\rightarrow$lan, IPTV$\rightarrow$zařízení
a naopak povolují IPTV$\rightarrow$IPTV.

\subsection{Systém}
Pro správnou funkci IPTV je potřeba zakázat IGMP snooping při startu routeru.
Toho docílíte následujícím řádkem, umístěným do souboru \uv{/etc/rc.local}:\\

\begin{lstlisting}
echo "0" > /sys/devices/virtual/net/br-IPTV/bridge/multicast_snooping
\end{lstlisting}

\subsection{Úryvky konfiguračních souborů}
\begin{figure}
	\lstinputlisting{figure/rc_local}
	\caption{Soubor \uv{/etc/rc.local}}
	\label{o2:frag:rclocal}
\end{figure}
\begin{figure}
	\lstinputlisting{figure/network}
	\caption{Úryvek ze souboru \uv{/etc/config/network}}
	\label{o2:frag:network}
\end{figure}
\begin{figure}
	\lstinputlisting{figure/firewall}
	\caption{Úryvek ze souboru \uv{/etc/config/firewall} -- VDSL uplink}
	\label{o2:frag:firewall:vdsl}
\end{figure}
\begin{figure}
	\lstinputlisting{figure/firewall2}
	\caption{Úryvek ze souboru \uv{/etc/config/firewall} -- další pravidla}
	\label{o2:frag:firewall:iptv}
\end{figure}
