\subsection{VDSL}
\label{net:vdsl}
Router TP-Link TD-W8970B je osazen i DSL modemem\footnote{
Konkrétně se jedná o Lantiq\footnote{Značka Lantiq dnes vystupuje jako součást značky Intel.}
XWay VRX268 @500MHz.} podporujícím jak \gls{adsl} 2+,
tak \gls{vdsl} 1/2. Na rozdíl od DSL routerů s \gls{soc} značky Broadcom
pro tento existují ovladače kompatibilní s \gls{openwrt} a tedy je možné tento
router použivat i pro DSL připojení. Pro zprovoznění DSL subsystému je potřeba
nainstalovat balíček \uv{ltq-vdsl-app} a stáhnout licenčně omezený binární
firmware modemu (procedura instalace binárního firmware blíže popsána v
sekci \ref{firstboot:vdsl})

Nastavení fyzické vrstvy modemu pak lze najít v souboru \uv{/etc/config/network}.
Pro toto nastavení slouží sekce typu vdsl. Příklad nastavení modemu pomocí
nástroje uci lze najít v ilustraci \ref{net:vdsl:phy}. Sekce obnáší následující
klíče:
\begin{description}
\item[annex] Jednoznakové označení DSL annexu, který se pro komunikaci má
použít (malým písmenem). Jednotlivé annexy jsou historicky-územně vymezeny
(důsledek technologií dříve používaných pro realizaci telefonní sítě). V
české republice se již dnes annex A nepoužívá a soudobá infrastruktura je
založena na annexu B.
\item[firmware] Cesta k souboru s binárním frmware pro modem. Pokud nemáte
závažný důvod směrovat ji jinam, měla by obnášet \uv{/lib/firmware/vdsl.bin}.
\item[tone] Vybírá sadu tónů, použitou pro navázání komunikace. Mezi možné
sady patří \uv{a}, \uv{b}, \uv{x}; přičemž
k sadám \uv{a} a \uv{b} lze připojit znak \uv{v}, značící vektorizaci.
Vektorizace je formou kompenzace přeslechů mezi páry telefonních vodičů,
která se musí předpočítat na \gls{dslam}u. Mimo jiné tvoří nedílnou součást VDSL2.
\item[xfer\_mode] Režim přenosu dat. Pro starší technologie (ADSL) bylo
používáno \uv{atm} (asynchronní přenosový mód), kdežto VDSL se opírá o \uv{ptm}
(paketový režim přenosu).
\end{description}
\begin{figure}
\begin{lstlisting}
uci set network.dsl.annex=b
uci set network.dsl.firmware=/lib/firmware/vdsl.bin
uci set network.dsl.tone=bv
uci set network.dsl.xfer_mode=ptm
uci commit network
\end{lstlisting}
\caption{Nastavení fyzické vrstvy \gls{vdsl} modemu pomocí uci}
\label{net:vdsl:phy}
\end{figure}

Samotné fyzické nastavení ale nepostačuje, je potřeba ještě nastavit přihlašovací
údaje. Tyto se nastavují pomocí odpovídajícího virtuálního rozhraní (zpravidla
pojmenovaného \uv{wan}). Toto rozhraní pak musí mít za fyzické rozhranní zvoleno
zařízení modemu (typicky \uv{ptm0}) s identifikátorem VLAN dle instrukcí poskytovatele
připojení (obdoba VPI/VCI u \gls{adsl}). Příkladem budiž jméno rozhraní
\uv{ptm0.848}. Toto rozhraní takřeka jistě nebude v nabídce, tudíž je potřeba
při jeho zadávání ve webovém rozhraní použít textové pole. Ukázku nastavení
pro síť O2 lze najít v sekci \ref{o2:frag:network}.
