\section*{Úvod}
Zdravím, tento neoficiální návod k \gls{openwrt} vznikl primárně proto,
aby novým uživatelům pomohl si zvyknout na trochu jiný firmware, než běžně
potkají. Neklade si za cíl vysvětlit, ani naučit vše, ale snad bude
stačit k většině běžných úkonů.

\subsection*{Co to je \gls{openwrt}?}
\gls{openwrt} je alternativní firmware pro síťový hardware, založený na
operačním systému Linux. Pokud Linux neznáte a slyšeli jste o něm jen
zkazky, že jde o nějakou zastaralou hrůzu, připomínající největší temnoty
systému MS DOS, pak jim nevěřte. I když to není na první pohled znát, i
takový Android je jen sadou aplikací, spuštěných v Linuxu. Protože -- stejně
stejně jako Linux -- je bezplatně spoluvyvíjen komunitou uživatelů a programátorů,
je množství podporovaných zařízení poněkud omezeno. Naštěstí ale máte jedno
ze zařízení, která podporována jsou.

\subsection*{Ovládání}
Co tedy potřebujete vědět o ovládání svého routeru?
\begin{description}
\item{\textbf{Webové rozhraní}}
Webové rozhraní dodávané s \gls{openwrt} se jmenuje \acrshort{luci}
(\acrlong{luci}). Je naprogramováno v jazyce Lua a existuje pro něj řada
rozšiřujících aplikací. Protože je ale \gls{openwrt} čistokrevný Linux
-- a tedy na něm lze provozovat prakticky jakoukoliv Linuxovou aplikaci --
je také současně nemožné naprogramovat webové rozhraní pro každou aplikaci.
\item{\textbf{Terminálové rozhraní}}
Síla (a pro nové uživatele také slabina) ovládání Linuxu spočívá v příkazové
řádce. Na tu se lze připojit pomocí zabezpečené služby \gls{ssh}. Ta probíhá
po šifrovaném komunikačním kanálu, což je jeden ze základních stavebních
kamenů bezpečnosti Vašeho routeru. Protože ale ssh potřebuje pro své fungování
krom uživatelského jména i heslo, není zprvu možné ssh použít. Pro nastavení
hesla tedy použijte webové rozhraní.
\end{description}

\subsection*{Seznam zkratek}
\printglossaries
