\subsection{Firmware VDSL}
\label{firstboot:vdslfw}
Nyní tedy ke zprovoznění samotného VDSL.
K tomu budete
potřebovat z internetu stáhnout soubor s firmware\footnote{
\url{https://www.telekom.de/hilfe/downloads/firmware-speedport-w921v-1.40.000.bin}},
%\url{http://hilfe.telekom.de/dlp/eki/downloads/Speedport/Speedport\%20W\%20921V/Firmware\_Speedport\_W921V\_1.40.000.bin}},
% http://www.telekom.de/hilfe/downloads/firmware-speedport-w921v-1.40.000.bin
který z licenčních důvodů není možné dodávat spolu s \gls{openwrt}\footnote{
Je možné, že soubor bude na webu vystaven v jiné verzi -- pak je potřeba tuto
stáhnout a pracovat s přejmenovanou kopií.
}. Soubor stáhněte, změňte číslo verze z 1.40.000 na 1.21.000 \footnote{\gls{openwrt}
\wrtversion počítá se starší verzí firmware VDSL subsystému.}
a nakopírujte programem WinSCP\footnote{Na Linuxu použijte příkaz
{\texttt{scp soubor.bin root@192.168.1.1:/tmp}}.} na svůj router.
Tento soubor uložte na router do adresáře \uv{/tmp}.

Nyní se na přihlaste na router klientem protokolu \gls{ssh} -- pod Windows můžete
použít program \uv{putty}\footnote{\url{http://www.slunecnice.cz/sw/putty-cz/}} -- viz \todo{ilustrace}\footnote{\texttt{ssh root@192.168.1.1}}.
Uživatel pro přihlášení
je \uv{root}, adresa hosta \uv{192.168.1.1}, heslo pak Vámi čerstvě nastavené.
Po přihlášení spusťte příkaz \uv{vdsl\_fw\_install.sh}, který ze staženého firmware
získá firmware pro VDSL subsystém modemu. Pro rozbalení je potřeba odsouhlasit
licenci staženého firmware.

Pro nastavení VDSL připojení nyní postupujte podle návodu v sekci \ref{net:vdsl}.
