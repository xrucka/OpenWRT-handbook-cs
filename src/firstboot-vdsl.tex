\subsection{Firmware VDSL}
\label{firstboot:vdslfw}
Nyní tedy ke zprovoznění samotného VDSL.
K tomu budete
potřebovat z internetu stáhnout soubor s firmware\footnote{
\url{https://www.telekom.de/hilfe/downloads/firmware-speedport-w921v-1.40.000.bin}},
%\url{http://hilfe.telekom.de/dlp/eki/downloads/Speedport/Speedport\%20W\%20921V/Firmware\_Speedport\_W921V\_1.40.000.bin}},
% http://www.telekom.de/hilfe/downloads/firmware-speedport-w921v-1.40.000.bin
který z licenčních důvodů není možné dodávat spolu s \gls{openwrt}\footnote{
Je možné, že soubor bude na webu vystaven v jiné verzi -- pak je potřeba tuto
stáhnout a pracovat s přejmenovanou kopií.
}. Soubor stáhněte a nakopírujte programem WinSCP\footnote{Na Linuxu použijte příkaz
{\texttt{scp soubor.bin root@192.168.1.1:/tmp}}.} na svůj router.
Tento soubor uložte na router do adresáře \uv{/tmp}.

Nyní se na přihlaste na router klientem protokolu \gls{ssh} -- pod Windows můžete
použít program \uv{putty}\footnote{\url{http://www.slunecnice.cz/sw/putty-cz/}} -- viz \todo{ilustrace}\footnote{\texttt{ssh root@192.168.1.1}}.
Uživatel pro přihlášení
je \uv{root}, adresa hosta \uv{192.168.1.1}, heslo pak Vámi čerstvě nastavené.
Po přihlášení spusťte následující příkazů, která pozáplatuje instalační skript
pro firmware a následně firmware nainstaluje\footnote{\gls{openwrt} 15.05.1
předpokládá starší, již nedostupnou, verzi firmware.}.
\begin{verbatim}
sed -i 's#Firmware_Speedport_W921V_1.21.000.bin#firmware-speedport-w921v-1.40.000.bin#g' /sbin/vdsl_fw_install.sh
sed -i 's#hilfe.telekom.de/dlp/eki/downloads/Speedport/Speedport%20W%20921V#www.telekom.de/hilfe/downloads#g' /sbin/vdsl_fw_install.sh
sed -i 's#0a099d08dbf091c74d685b532cbb1390#409a69b9a4eeffd681cb2dd84d6edf6d#g' /sbin/vdsl_fw_install.sh
sed -i 's#59dd9dc81195c6854433c691b163f757#655442e31deaa42c9c68944869361ec0#g' /sbin/vdsl_fw_install.sh
sed -i 's#06b6ab3481b8d3eb7e8bf6131f7f6b7f#57f2d07f59e11250ce1219bad99c1eda#g' /sbin/vdsl_fw_install.sh
ls -s /tmp/firmware-speedport-w921v-1.40.000.bin /tmp/Firmware_Speedport_W921V_1.21.000.bin#firmware-speedport-w921v-1.40.000.bin
vdsl_fw_install.sh
\end{verbatim}
Pro rozbalení je potřeba odsouhlasit licenci staženého firmware.

Pro nastavení VDSL připojení nyní postupujte podle návodu v sekci \ref{net:vdsl}.
