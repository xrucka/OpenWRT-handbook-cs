\section{Síť}
Nastavení sítě najdete ve webovém rozhraní v záložce \uv{Network}. Zde,
na kartě \uv{Interfaces} je k dispozici přehled síťových rozhraní, která
jsou na routeru nastavena.

\gls{openwrt} pracuje s konceptem virtuálních rozhraní, která jsou mapována
na rozhraní fyzická. Toto umožňuje jednomu fyzickému rozhraní nastavit více
různých adres. V případě, že nějaké virtuální rozhraní používáme k přidání
adresy, slouží jako jeho fyzické rozhraní jméno virtuálního rozhraní, kterému
přidává adresu. Toto jméno je na začátku předsazeno znakem zavináče. Pro
ilustraci se zaměřte na rozhraní \uv{wan} a \uv{wan6}.

\paragraph{Upozornění:} tlačítko \uv{Delete} provádí operaci \uv{Save \& Apply},
buďte tedy při práci s ním nejvýše opatrní. Pokud budete provádět změny v
nastavení rozhraní LAN, pak si raději vytvořte pomocné rozhraní se statickou
adresou, než abyste si odřízli přistup k routeru.

\subsection{Úprava existujícího rozhraní}
Úpravu existujícího nastavení provedete tlačítkem \uv{edit} u odpovídajícího
síťového rozhraní. Tímto se dostanete k volbám tohoto rozhraní, kde můžete
na záložce \uv{General setup} nastavit IP adresy tohoto rozhraní.

Na kartě \uv{Physical settings} pak nastavujete na která fyzická rozhraní
se má virtuální rozhraní promítat\footnote{Pomocí jména fyzického rozhraní
\uv{@jmenorozhrani} nastavujete doplnění konfigurace virtuálního rozhraní
s názvem \uv{jmenorozhrani}.} Pokud chcete použít stejnou adresaci pro dvě
různá fyzická rozhraní, musí být nad těmito rozhraními vytvořen síťový
bridge. Bridge vytvoří nad síťovým rozhraním virtuální switch, kde
jednotlivá fyzická rozhraní představují porty switche. Toto má za důsledek
to, že počítače z jednoho fyzického rozhraní budou moci komunikovat přímo
s počítači z druhého fyzického rozhraní. Stejně, jako kdyby byly připojeny
do skutečného switche. Řada routerů navíc má vestavěn hardwarový switch,
pokrývající ethernetové porty. Pokud máte ethernetové rozhraní \gls{wan},
nikdy jej nepřidávejte do bridge s \gls{lan} bez ověření, zda nejsou na
stejném hardwarovém switchi.

Na kartě \uv{Firewall} máte k dispozici možnost přiřadit rozhraní do zóny
firewallu. Vlastní firewall pak najdete v samostatné záložce. Pro jeho fungování
je potřeba přijmout, že openwrt používá tzv. zónový firewall - místo definice
pravidel pro jednotlivá fyzická rozhraní definujete pravidla pro zóny.
Konfigurací firewallu tato příručka nepokrývá, s výjimkou některých nastavení
specifických pro síť O2, uvedených na konci příručky.

\subsection{Switch}
\label{net:switch}
V záložce \uv{Network $\rightarrow$ Switch} najdete nastavení vestavěného
hardwarového switche. Tento má tzv. logické porty, jejichž číslování nemusí
odpovídat číslování fyzickému. A mnohdy také neodpovídá. U některých zařízení
jsou LAN a WAN porty prezentovány jako oddělené hardwarové síťové karty
(kde switch je připojen k LAN), na jiných jsou obé jen jinak barevně rozlišené
porty na zádech routeru, všechny připojené do switche. Aby tato rozhraní v
systému vystupovala odděleně, je každému z nich nastavena oddělená \gls{vlan}.
\begin{table}[h!]
\begin{center}
\begin{tabular}{|l|l|}
\hline
Číslo portu switche & fyzické označení portu \\ \hline
0 & LAN 2  \\ \hline
1 & nezapojeno \\ \hline
2 & LAN 3 \\ \hline
3 & nezapojeno \\ \hline
4 & LAN 4 \\ \hline
5 & LAN 1 \\ \hline
6 & CPU \\ \hline
\end{tabular}
\caption{Mapování logických portů switche na fyzické označení portu pro TP-Link TD-W8970B}
\end{center}
\end{table}


Co je to \gls{vlan}? Virtual LAN - způsob značkování
paketů na fyzické lince, umožňující přenos více logických sítí po stejném
médiu. Aby \gls{vlan} fungovala, musí se při vysílání paketu přidat před
paket značka s číslem, do které \gls{vlan} paket patří. Příjemce
pak musí mít pro stejnou \gls{vlan} nastavenu stejnou síť. Protože ne vždy
je toto značkování (angl. tagged) žádoucí, je možné použít neznačkovaný režim
portu. To znamená, že při vysílání ven je značka odstraněna a naopak při
přijetí je značka doplněna (angl. untagged). Port může být veden jako neznačkovaný
pro nejvýše jednu VLAN.

Toto umožňuje realizovat jak WAN, tak LAN rozhraní jedním switchem. Logické
\gls{wan} rozhraní je tvořeno jednou \gls{vlan} (např. VLAN ID 2), kdežto
LAN jinou \gls{vlan} (např. VLAN ID 1). Porty, patřící do fyzické LAN jsou
pak neznačkované (untagged) pro \gls{vlan} 1 se zakázanou \gls{vlan} 2, porty
pro \gls{wan} pak naopak.

Některé routery mají hybridní zapojení síťového hardware -- \gls{wan} i
\gls{lan} jsou přivedeny do stejného switche, kde jsou realizovány pomocí \gls{vlan}.
Avšak namísto toho, aby byl tento switch dalším portem připojen k CPU, je k CPU
připojen hned dvěma porty. Jeden tento port pak patří do \gls{vlan} reprezentující
\gls{wan}, druhý pak plní stejnou úlohu pro \gls{lan}.

Pokud byste chtěli \gls{wan} port přemostit do \gls{lan}, pak si zkontrolujte
že není reprezentován jako \gls{vlan}. Přemostění dvou vlan pomocí bridge
by mohlo vyústit v chybnou konfiguraci routeru. V takovém případě raději přidejte
\gls{wan} port do stejné \gls{vlan}. Naopak, pokud budete chtít některý z LAN
portů použít pro WAN, přiřaďte mu (neznačkovanou) \gls{vlan} použitou pro WAN
port.

Ačkoliv \gls{openwrt} umožňuje mít odlišné VLAN ID a identifikátor pro síťové
rozhraní, je lepší mít tyto nastaveny na stejnou hodnotu -- odlišné nastavení
nemusí na řadě routerů fungovat. Na konkrétní \gls{vlan} se pak V konfiguraci
rozhraní odkážete přes jméno switche (zpravidla \uv{eth0}), doplněné tečkou
a číslem rozhraní (tedy \uv{eth0.1} pro LAN). Aby tato možnost fungovala, musí
být port CPU vždy označen za značkovaný pro každou \gls{vlan}.

Se změnami VLAN buďte opatrní, zvláště pokud zasahujete do připojení k procesoru.
V takovém případě doporučuji jednak použít externí úložiště, druhak nastavit
vše najednou pomocí příkazové řádky (nejlépe textovým editorem), vše si po sobě
několikrát opakovaně přečíst a teprve posléze nastavení aplikovat restartem.

\subsection{VDSL}
\label{net:vdsl}

