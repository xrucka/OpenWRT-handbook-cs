\section{Jak funguje nastavení}
Nastavení \gls{openwrt} lze (obecně) spravovat dvěma způsoby -- webovým
rozhraním a příkazovou řádkou. Každému uživateli může vyhovovat jiné ovládání,
v obecné rovině ale platí že s příkazovou řádkou zmůžete více. Na základní
ovládání ale plně postačuje webové rozhraní.

Webové rozhraní má dva režimy ukládání změn -- \uv{Save} a \uv{Save \& Apply}.
Režim \uv{Save} pouze poznačí změny v nastavení, avšak neaplikuje je. Pokud
není k dispozici tlačítko \uv{Save \& Apply}, můžete očekávat že změny budou
aplikovány obratem (např. některá nastavení síťových rozhraní). Režim
\uv{Save \& Apply} je pak explicitně rozlišen tam, kde je možné provádět změny
bez jejich okamžité aplikace (\uv{Save}), nebo je naopak žádoucí explicitně
chtít jejich aplikaci (\uv{Save \& Apply}).

Při použití příkazové řádky najdete nastavení základního systému v adresáři
routeru \uv{/etc/config}.
Zde se nachází sada souborů, kde každý soubor odpovídá nějakému subsystému.
S těmito soubory můžete manipulovat buď textovým editorem, nebo pomocí nástroje
\uv{uci}. Soubory jsou dále strukturované na jednotlivé sekce, které obsahují
klíče. Spojením jména subsystému, sekce a klíče pak vzniká plnohodnotný
název klíče. Některé klíče se v téže sekci mohou opakovat, jiné nesmí.

Základní operace s \uv{uci} jsou:
\begin{description}
\item[\textbf{show}] -- příkaz \uv{uci show wireless} vypíše nastavení wifi.
\item[\textbf{set}] -- příkaz \uv{uci set 'wireless.@wifi-iface[0].ssid=OpenWRT-je-doma'}
změní jméno wifi sítě na řetězec \uv{OpenWRT-je-doma}. Apostrofy v příkazu jsou nutné
kvůli speciálním znakům [~] ve jméně klíče. Odpovídá změně uložené tlačítkem
\uv{Save} webového rozhraní.
\item[\textbf{commit}] -- aplikuje změny v nastavení. Odpovídá aplikaci změn,
provedené tlačítkem \uv{Save \& Apply} webového rozhraní.
\item[\textbf{--help}] -- vyobrazí nápovědu k ovládání nástroje uci.
\end{description}
